 %%%%%%%%%%%%%%%%%%%%%%%%%%%%%%%%%%%%%%%%%%%%%%%%%%%%%%%%%%%%%%%%%%%%%%%%%%%%%%
%                             Curriculum Vitae                               %
%%%%%%%%%%%%%%%%%%%%%%%%%%%%%%%%%%%%%%%%%%%%%%%%%%%%%%%%%%%%%%%%%%%%%%%%%%%%%%
\documentclass[10pt,a4]{article}
\topmargin-2.0cm
\advance\oddsidemargin-1.2cm
\advance\evensidemargin-1.2cm
\textheight9.22in
\textwidth6.4in
\newcommand\bb[1]{\mbox{\em #1}}
%\def\baselinestretch{1.25}
\def\baselinestretch{1.0}

\usepackage{multicol}
% The use of the times package forces the use of the type-1 times
% roman font, but the times roman font does not look nice.
% Besides the times roman font still does not print correctly on
% the dopy printer.
%\usepackage{times}

\usepackage{fancyhdr}
\usepackage{origpagecounting}
\usepackage[dvips]{color}
\usepackage{tabularx}
\usepackage{hyperref}
\usepackage{soul}
\usepackage[T1]{fontenc}

\newcounter{myEnumCounter}
\newcounter{mySaveCounter}
\renewenvironment{enumerate}{%
  \begin{list}{\arabic{myEnumCounter}.}{\usecounter{myEnumCounter}%
  \setcounter{myEnumCounter}{\value{mySaveCounter}}}
  }{%
  \setcounter{mySaveCounter}{\value{myEnumCounter}}\end{list}%
}
\newcommand\myEnumReset{\setcounter{mySaveCounter}{0}}

% The old enumerate environment is rewritten, so you need no special command to
% start continuing counting. With the command \myEnumReset you can Reset the couter
% at any place in the text.

% http://www.educat.hu-berlin.de/~voss/lyx/list/enum.phtml

\definecolor{gray}{rgb}{0.4,0.4,0.4}

\begin{document}

%\thispagestyle{empty}
%\pagestyle{plain}

\thispagestyle{fancy}
%\pagenumbering{gobble}
%\fancyhead[location]{text}
% Leave Left and Right Header empty.
\lhead{\textcolor{gray}{\it Junghyun Lee}}
%\rhead{\textcolor{gray}{\thepage/\totalpages{}}}
%\rhead{\thepage}
\renewcommand{\headrulewidth}{0pt}
\renewcommand{\footrulewidth}{0pt}
\fancyfoot[C]{\footnotesize \textcolor{gray}{}}
%A copy of this curriculum vitae, publications and 
%talk slides are available for download at
%http://www.stanford.edu/$\sim$sundaes/application}}


%\pagestyle{myheadings}
%\markboth{Sundar Iyer}{Sundar Iyer}

\vspace*{0.4cm}
\begin{center}
{\huge \bf Junghyun Lee}
\vspace*{0.25cm}
\end{center}

\begin{small}

%===================================
\begin{tabbing}
\=xxxxxxxx\=xxxxxxxx\=xxxxxxxx\=\kill
\begin{tabular*}{\linewidth}{l@{\extracolsep{\fill}}r}

291 Daehak-ro & Phone: (+82)10 5819-2684 \\
Yuseong-gu, Daejeon 34141 &  Email: jh\_lee00 (AT) kaist.ac.kr \\
%https://sites.google.com/view/junghyunlee
 & Alt: nick.jhlee00 (AT) gmail.com \\
 Github: \url{https://github.com/nick-jhlee}
 & Personal website: \url{https://nick-jhlee.netlify.app/}
\end{tabular*}
\end{tabbing}

\vspace*{0.2cm}


%==========================================
%\vspace{0.20cm}

\subsection*{PARTICULARS}

%{\color{DarkSeaGreen}} \hrule
\hrule
\vspace{0.2cm}
%%%%%%%%%%%%%%%%%%%%%%%%%%%%%%%

\subsubsection*{EDUCATION}
%\vspace{0.2cm}

%\begin{tabbing}
%xxxxxxxx\=xxxxxxxx\=xxxxxxxx\=xxxxxxxx\=\kill
%\>{\bf Academic record}\\[.7em]

%\>\begin{tabular}{|l|l|l|}
%\hline
%Certificate	&	Place of study	&	Year\\
%\hline
%Ph. D. in Computer Sc.  & Stanford University     & {\it 2000-, Defended 2003} \\
%\hline
%M. S. in Computer Sc. & Stanford University     & {\it June 2000} \\
%\hline
%B. Tech in Computer Sc. and Engg. & I.I.T. Bombay  &	{\it April 1998} \\
%\hline
%\end{tabular}
%\end{tabbing}

%xxxx\=xxxxxxxx\=xxxxxxxx\=xxxxxxxx\=\kill

\begin{tabbing}
xxxx\=xxxxxxxx\=xxxxxxxx\=xxxxxxxx\=\kill
%\>\begin{tabular*}{6.1in}{lr}

\>\begin{tabular*}{0.9\linewidth}{l@{\extracolsep{\fill}}r}
%Korea Advanced Institute of Science and Technology (KAIST) & Seoul, ROK \\
%MS in Graduate School of AI & {\it August 2023 (expected)}\\
%Advisor: Prof. Se-Young Yun, Graduate School of AI \\
%& \\
	
Korea Advanced Institute of Science and Technology (KAIST) & Daejeon, ROK \\
BS in Mathematical Sciences, Computer Science({\it Double Major}) & {\it August 2021 (expected)}\\
%Academic advisor: Prof. Soonsik Kwon, Department of Mathematical Sciences \\
%{\it Cumulative GPA: 3.82 / 4.3, Major GPA: 3.78 / 4.3} \\
%	(Cum laude)} \\
 & \\
 
Changwon Science High School (CSHS) & Changwon, ROK \\
{\it Early graduation} & {\it March 2017}
\end{tabular*}
\end{tabbing}

\subsubsection*{CURRENT STATUS}
\begin{list}{}{}
\item Citizen of Republic of Korea (ROK).
\end{list}

\subsubsection*{ACADEMIC INTERESTS}
%\hrule
%\vspace{0.2cm}

\begin{itemize}{}{}
\item (Statistical) Machine Learning

\item Deep Learning

\item Related mathematical theories (e.g. Probability Theory, Optimization, Statistics)

\item Various applications of ML/DL

\item "Other" mathematics (graph theory, discrete geometry, algorithms...etc.)
\end{itemize}
%\vspace{0.1cm}
%
%\newpage
%\pagestyle{fancy}
%\lhead{\textcolor{gray}{\it Junghyun Lee}}
%%\rhead{\textcolor{gray}{\thepage/\totalpages{}}}
%\fancyfoot[C]{}
%\subsubsection*{RELEVANT COURSEWORK}
%
%\vspace{0.2cm}
%
%\begin{tabularx}{\linewidth}{ @{} X X @{} }
%	{\bf Dept. of Mathematical Sciences}
%  \begin{itemize}
%	\item Analysis I/II
%	\item Complex Function Theory$^{(*)}$
%	\item Linear Algebra
%	\item Modern Algebra I
%	\item Topology
%	\item Mathematical Statistics
%%	\item Advanced Statistics$^{(**)}$
%%	\item Elementary Probability Theory$^{(**)}$
%%	\item Lebesgue Integral Theory
%%	\item Random Matrix Theory and Its Application$^{\#, (*)}$
%	\item Introduction to Numerical Analysis
%%	\item Numerical Analysis
%	\item Discrete Mathematics
%	\item Discrete Geometry
%	\item Topological Methods in Combinatorics
%	\item Matroid Theory
%%	\item Probabilistic Methods
%	\item (Enumerative) Combinatorics$^{(*), (**)}$
%	\item Introduction to Mathematical Biology
%  \end{itemize} &
%	
%	{\bf School of Computing}
%  \begin{itemize}
%	\item Machine Learning
%	\item Introduction to Artificial Intelligence
%	\item Data Structures
%	\item Introduction to Algorithms
%	\item Design and Analysis of Algorithm$^{(*)}$
%	\item Programming Languages
%	\item Introduction to Logic for Computer Science
%  \end{itemize}
%\end{tabularx}
%
%\begin{tabularx}{\linewidth}{ @{} X X @{} }
%	{\bf Dept. of Electrical Engineering}
%  \begin{itemize}
%	\item Statistical Learning Theory$^{(*)}$
%  \end{itemize} &
%	
%	{\bf Dept. of Physics}
%  \begin{itemize}
%	\item Physical Mathematics I
%  \end{itemize}
%\end{tabularx}
%
%
%(*: graduate-level course)
%(**: audited course)
%(\#: currently registered)


\subsection*{ACADEMIC HONORS}
\hrule
\vspace{0.2cm}
\begin{itemize}
%==================

%\item Cum laude

\item Spring Semester Freshmen Dean's List, 2017.

\end{itemize}

\pagestyle{fancy}
\lhead{\textcolor{gray}{\it Junghyun Lee}}
%\rhead{\textcolor{gray}{\thepage/\totalpages{}}}
\fancyfoot[C]{}


%\vspace{0.1cm}

\subsection*{RESEARCH EXPERIENCE}
\hrule
\vspace{0.2cm}
\begin{itemize}
  
 \item 
% {\bf [With Cheolhyeong Lee$^{**}$]
 	{\bf [Alone] Undergrad Research: OSI Lab}, Spring 2020 - Present. \\
 {\bf Supervisor:} Prof. Se-Young Yun (Graduate School of AI, KAIST) \\
 {\bf Research topic:} {\it Toward a Better Understanding of Dynamics of Deep Neural Networks and SGD } \\
% ($**$ Currently Post-doctoral associate of Center for Data Science at NYU)\\

%  \item {\bf [Alone] Undergrad Research:
%	% 	 Artificial Intelligence \& Machine Learning Lab}
%	AIM Laboratory}, Summer 2020 - Present. \\
%{\bf Supervisor:} Prof. Chang Dong Yoo, Prof. Gwangsu Kim (Dept. of Electrical Engineering, KAIST) \\
%{\bf Research topic:} {\it Fair Graph Analysis (TBD) } \\

 \item {\bf [Alone] Undergrad Research:
% 	 Artificial Intelligence \& Machine Learning Lab}
  AIM Lab}, Winter 2019 - Present. \\
 {\bf Supervisor:} Prof. Chang Dong Yoo, Prof. Gwangsu Kim (Dept. of Electrical Engineering, KAIST) \\
 {\bf Collaborator:} Matt Olfat (Citadel) \\
 {\bf Research topic:} {\it Fair PCA via Optimization over Stiefel Manifold} \\
% ($*$ Citadel)\\

 \item {\bf [With Chani Jung, Yoo Hwa Park, Dongmin Lee] Undergrad Research: COINSE Lab}, Winter 2021 - Present. \\
{\bf Supervisor:} Prof. Shin Yoo (School of Computing, KAIST) \\
{\bf Research topic:} {\it SWAY for Decision Space of Permutations with Case Study on Test Case Prioritisation } \\
% ($**$ Currently Post-doctoral associate of Center for Data Science at NYU)\\

 \item {\bf [Alone] Individual Study}, Summer 2019 - Fall 2019. \\
 {\bf Supervisor:} Prof. Andreas Holmsen (Dept. of Mathematical Sciences, KAIST) \\
 {\bf \st{Research} Seminar topic 1:} {\it Asymptotics for the number of $C_4$'s in a graph under certain condition}, \\
 {\bf \st{Research} Seminar topic 2:} {\it Maximum number of columns in a 0-1 $2n \times n$ matrix with no induced $2 \times 2$ identity matrix} \\


 \item {\bf [With Seokmin Ha] MAS480(B): Introduction to Mathematical Biology}, Fall 2018. \\
 {\bf Supervisor:} Prof. Jaekyung Kim (Dept. of Mathematical Sciences, KAIST) \\
 {\bf Research topic:} {\it Reverse Analysis Problem of Two-gene System in the
Perspective of Adaptation} \\
% {\bf Abstract:} In this work, we consider a reverse analysis problem regarding two-gene system. Using the mixed-integer dynamic optimization framework that has been presented in previous works, we extend the research of finding the design principles of three-gene system in perspective of adaptability to two-gene system. 
% We show that the negative of amplitude and area of response are in a weak trade-off relation by computing the Pareto front for those two objective functions. We also find that the majority of the Pareto optimal circuits have the common structure of amplified negative feedback topology.
% Lastly, we deduce that the conditions for a two-gene system to achieve perfect adaptation is to have the characteristics of high sensitivity and low speed.

\item {\bf CSHS Mathematics Research and Education Program (R\&E)}, Mar 2015 - Feb 2017. \\
 {\bf Supervisor:} Mr. Seungkyun Cha (Mathematics Department, CSHS), Prof. Jisoo Byun (Dept. of Mathematics Education, Kyungnam University) \\
 {\bf Research topic:} {\it Some Loci in the Animation of a Sangaku Diagram} \\
% {\bf Abstract:} In a symmetric partition of a regular n-gon into n congruent subtriangles and a regular n-gon in the center, we determine the loci of the incenter and points of tangency of the incircle a subtriangle.
 
\end{itemize}

% List of papers and publications and patents ?
%===========================================
%\vspace{0.1cm}
\subsection*{PUBLICATIONS}
\hrule
\vspace{0.2cm}

\subsubsection*{WORKING/PENDING PAPERS}
\begin{enumerate}
	\item
	\textbf{Junghyun Lee}, Gwangsu Kim, Chang D. Yoo, Matt Olfat.
	``Fair PCA via Optimization over Stiefel Manifold" (Work in progress, Unpublished)

%	\item
%\textbf{Junghyun Lee}, Se-Young Yun.
%``Effect of Weight Ranks on Training Dynamics of Deep Neural Networks" (Work in progress, Unpublished)
\end{enumerate}

\myEnumReset
\subsubsection*{PREPRINTS, TECHNICAL REPORTS}
\begin{enumerate}
	\item (to be filled in...)
\end{enumerate}

\myEnumReset
\subsubsection*{JOURNAL}
\begin{enumerate}
    \item 
\textbf{Junghyun Lee}, Minyoung Hwang, Cheolwon Bae.
``Some Loci in the Animation of a Sangaku Diagram",
{\it Forum Geometricorum}, 2016, vol. 16, pp. 187-191.
\end{enumerate}

\myEnumReset
\subsubsection*{PEER-REVIEWED CONFERENCE}
\begin{enumerate}
	\item (to be filed in)
\end{enumerate}	



\subsection*{COURSEWORKS}
\hrule
\vspace{0.2cm}

\subsubsection*{PROJECTS}
\begin{itemize}
	
	\item {\bf [With Chani Jung, Yoo Hwa Park, Dongmin Lee] CS454: Artificial Intelligence based Software Engineering}, Fall Semester, 2020. \\
	Proposes a way to apply SWAY to SBSE problems whose decision space consists of permutations.
	
	\item {\bf [With Minyoung Hwang, Junseok Choi] CS492: Deep Learning for Real-Life Problems}, Fall Semester, 2020. \\
	Proposes deep learning based solution for semi-supervised classification on Naver Fashion Dataset, and Korean Open-Domain QA task on Naver KorQuAD-Open dataset. ($2$nd, $1$st place in leaderboard)
	
	\item {\bf [With Youngjin Jin, Minsung Park, Hyunjin Kim] CS376: Machine Learning}, Fall Semester, 2018. \\
	Built a predictive model for given set of (training) data. (cf. Gotham city's apartment prices) 
	% Given a large set of training data, a model was made to predict the apartment prices.
	
	\item {\bf [With Youngjin Jin, Minsung Park] CS470: Introduction to Artificial Intelligence}, Fall Semester, 2019. \\
	Implemented a model for music genre classification problem.
	% Implemented the {\it Spherical CNN} as proposed by Cohen \& Welling and applied it to a different dataset. {\bf to be updated...}
	
	\item {\bf [Alone] CS300: Intro. to Algorithms}, Spring Semester, 2018. \\
	Implement a solver for the game 'Bloxorz'.
	
	\item {\bf [With Seokmin Ha] MAS480(B): Introduction to Mathematical Biology}, Fall Semester, 2018. \\
	(See below for details)
	
	\item {\bf [With Hyunjin Kim] CS330: Operating Systems and Lab}, Spring Semester, 2019. \\
	Built Pintos: a small operating system.
	
\end{itemize}

\myEnumReset
\subsubsection*{REPORTS}
\begin{enumerate}
	\item {\bf Junghyun Lee}, Chani Jung, Yoo Hwa Park, Dongmin Lee.
	``SWAY for Decision Space of Permutations with Case Study on Test Case Prioritisation",
	{\it CS454: Artificial Intelligence Based Software Engineering}, 2020 Fall.
	(Work in progress)
	
	\item Seokmin Ha, {\bf Junghyun Lee}. 
	``Reverse Analysis Problem of Two-gene System in the Perspective of Adaptation``,
	{\it MAS480(B): Topics in Mathematics <Introduction to Mathematical Biology>}, 2018 Fall.
	
	\item {\bf Junghyun Lee}. 
	``Critical Review on Theoretical Aspects of Binary Decision Diagram, with a Focus in Variable Ordering",
	{\it CS402: Introduction to Logic for Computer Science}, 2020 Spring.
	
	\item Junseok Choi, Minyoung Hwang, {\bf Junghyun Lee}
	``Semi-Supervised Learning Task on Naver Fashion Dataset",
	{\it CS492(I): Special Topics in Computer Science <Deep Learning for Real-World Problems>}, 2020 Fall.
	
	\item Minyoung Hwang, Junseok Choi, {\bf Junghyun Lee}
	``Korean Open-Domain QA Task on Naver KorQuAD-Open Dataset",
	{\it CS492(I): Special Topics in Computer Science <Deep Learning for Real-World Problems>}, 2020 Fall.
	
\end{enumerate}



%\vspace{0.1cm}
\subsection*{TEACHING EXPERIENCE}
\hrule
\vspace{0.2cm}

\subsubsection*{TEACHING ASSISTANT}
\begin{itemize}

% \item {\bf TCIS-KAIST Winter Science Camp}, Taejon Christian International School \& KAIST Global Institute for Talented Education, Winter 2018, KAIST. \\
% (description)

\item {\bf HSS302: Special Lectures on Linguistics \textless Language Register and English\textgreater}, Prof. Seonmin Park, Spring 2018, KAIST.
%(description)

\item {\bf English Camp for Incoming Freshmen}, EFL Office, Jan 2019, KAIST.
%(description)

\item {\bf English Camp for Incoming Freshmen}, EFL Office, Jan 2018, KAIST.
%(description)

\end{itemize}

\subsubsection*{FRESHMEN TUTORING}
\begin{itemize}

\item {\bf MAS102: Calculus 2}, Fall 2018, KAIST.

\item {\bf MAS101: Calculus 1}, Spring 2018, KAIST.

\end{itemize}

\subsubsection*{UNOFFICIAL/VOLUNTARY TUTORING}
\begin{itemize}

\item {\bf MAS102, PH142, MAS109}, Fall 2017, KAIST. \\
with 10$\sim$15 freshmen taking the courses

\item {\bf MAS101, PH141, CH101, MAS109}, Spring 2017, KAIST. \\
with 10$\sim$15 freshmen taking the courses
%\item {\bf MAS212}, Spring 2019, KAIST. \\
%with 3$\sim$4 students taking the courses


\end{itemize}


%\subsection*{PROFESSIONAL ACTIVITIES}
%\hrule
%\vspace{0.2cm}

%\subsubsection*{PARTICIPANT OF TSUKUBA SCIENCE EDGE}
%(description)
%Tsukuba, Japan, 2016.

%\subsubsection*{PARTICIPANT OF ICISTS-2018}
%(description)

%\subsubsection*{THE-KAIST INNOVATION \& IMPACT SUMMIT STUDENT AMBASSADOR}
%(description)KAIST, 2019
%\vspace{0.1cm}


\lhead{\textcolor{gray}{\it Junghyun Lee}}
%\rhead{\textcolor{gray}{\thepage/\totalpages{}}}
\fancyfoot[C]{}

\subsection*{SKILLS}
\hrule
\vspace{0.2cm}

\subsubsection*{Computer Skills}

\begin{itemize}
	\item Languages: \textbf{Python,
%	C,
%	Java,
%	Kotlin,
%	Scala
	}
	\item Applications : \textbf{Matlab,
%	 Mathematica,
	 LaTex}
\end{itemize}

\subsubsection*{LANGUAGE}

\begin{itemize}{}{}
	\item \textbf{Korean}: Native
	\item \textbf{English}: (Almost) Native \\
%	Lived in the United States (Feb. 2010 $\sim$ Feb. 2014) \\
	Mock TOEFL iBT 118 (2017)
	
\end{itemize}

\vspace{0.1cm}

%
%\subsection*{REFERENCES}
%\hrule
%\vspace{0.2cm}
%
%\subsubsection*{Se-Young Yun}
%
%\begin{itemize}{}{}
%	\item Assistant Professor, Graduate School of AI (KAIST)
%	\item OSI Lab
%\end{itemize}
%
%
%\subsubsection*{Chang Dong Yoo}
%
%\begin{itemize}
%	\item Professor, Dept. of Electrical Engineering (KAIST)
%	\item AIM Lab
%\end{itemize}
%
%\subsubsection*{Gwangsu Kim}
%
%\begin{itemize}{}{}
%	\item Research Professor, Dept. of Electrical Engineering, (KAIST)
%	\item AIM Lab
%\end{itemize}
%
%\vspace{0.1cm}


\subsection*{MISC.}
\hrule
\vspace{0.2cm}

\subsubsection*{KAIST ORCHESTRA}
\begin{itemize}
	\item First Violinist, Mar 2017 - Present.
	
	\item {\bf Principal First Violinist}, Jan 2018 - Dec 2018.
	
%	\item Concertmaster, Jan 2020 - Dec 2020.
\end{itemize}

\subsubsection*{ICISTS}
\begin{itemize}
	\item Division of Global Partnership, Sep 2018 - Aug 2019.
	
	\item {\bf TF leader} of {\it Opening/Gala Night} (ICISTS-2019)
	
	\item TF member of {\it Science in a Nutshell} (ICISTS-2019)
	
	\item {\bf Vice President}, Sep 2019 - Jul 2020.
	
%	\item {\bf TF leader} of {\it TED}x{\it KAIST}, Oct 2019 - Apr 2020
	
\end{itemize}

%\subsubsection*{TEDxKAIST}
%\begin{itemize}
%	\item {\bf Team Leader} of {\it TED}x{\it KAIST}, Oct 2019 - Present.
%	
%\end{itemize}

\subsubsection*{KAIST Mathematical Sciences Student Council}
\begin{itemize}
	\item Member of department student council, Mar 2018 - Present.
	
	\item In charge of {\it Mathematical Sciences Help-Desk} (Mar 2018 - June 2019) \\
	A short lecture series (given by selected math undergrad.) that takes place a week before the exam period to help all students with Basic Elective courses. (MAS109, MAS201, MAS250)
\end{itemize}

\newpage
\pagestyle{fancy}
\lhead{\textcolor{gray}{\it Junghyun Lee}}
%\rhead{\textcolor{gray}{\thepage/\totalpages{}}}
\fancyfoot[C]{}

%anything else?

\end{small}
\end{document}
%%%%%%%%%%%%%%%%%%%%%%%%%%%%%%%%%%%%%%%%%%%%%%%%%%%%%%%%%%%%%%%%%%%%%%%%
